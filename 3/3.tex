\documentclass[25pt]{article}
\usepackage[utf8]{inputenc}

\title{математическая статистика. Билеты}
\author{alexander.veselyev }
\date{\today}

\usepackage{natbib}
\usepackage{graphicx}
\usepackage[utf8]{inputenc} % указывает кодировку документа
\usepackage[T2A]{fontenc} % указывает внутреннюю кодировку TeX 
\usepackage[russian]{babel} % указывает язык документа

\usepackage{amsmath}

\usepackage{amssymb}
\usepackage{amsthm}

\newcommand\independent{\protect\mathpalette{\protect\independenT}{\perp}}
\def\independenT#1#2{\mathrel{\rlap{$#1#2$}\mkern2mu{#1#2}}}


\usepackage{color}   %May be necessary if you want to color links
\usepackage{hyperref}
\hypersetup{
    colorlinks=true, %set true if you want colored links
    linktoc=all,     %set to all if you want both sections and subsections linked
    linkcolor=blue,  %choose some color if you want links to stand out
}

\begin{document}


\section{Непараметрическая задача. Эмпирическое распределение и эмпирическая функция
распределения. Теорема Гливенко-Кантелли.}

\textbf{Определение} Пусть $X_1,\dots,X_n$ - выборка, $B \in \mathcal{B}(\mathbb{R})$\\
\textbf{эмпирическое распределение}: $$P_n^*(B) = \frac{\sum_{i=1}^n I(X_i \in B)}{n}$$
По УЗБЧ: $P_n^*(B) \xrightarrow{\text{п.н.}} P(B)$\\
А значит, \textbf{эмпирическая функция распределения}\\ $F_n^*(x) = P_n^*((-\infty, x]) \xrightarrow{\text{п.н.}} F(x)$ (теорема Гливенко)\\
\\
\textbf{Теорема}(Гливенко-Кантелли)\\
Пусть $X_1,\dots,X_n$ - выборка с функцией распределения $F(x)$\\
Определим $D_n = \sup_{x\in \mathbb{R}}|F_n^*(x) - F(x)|$\\
Тогда $D_n \xrightarrow{\text{п.н.}} 0$
\\
$\blacktriangle$
\\
$F_n^*(x,\omega)$ - функция распределения $\Rightarrow$ непрерыная справа $\Rightarrow$ функция $\forall \omega\ |F_n^*(x,\omega) - F(x)|$ непрерывна справа $\Rightarrow$ в $D_n$ супремум можно брать не по всем действительным точкам, а по всем рациональным (потому что действительные точки можем приблизить рациональными) $\Rightarrow$
$$D_n = \sup_{x\in\mathbb{Q}}|F_n^*(x,\omega) - F(x)|$$
является случайной величиной, как супремум счетного числа случайных величин.\\
Фиксируем $N \in \mathbb{N}$, положим $\forall k = \overline{1,N-1}$:\\
$x_k = \inf\{x\in\mathbb{R}:\ F(x) \geq \frac{k}{N}\}$ (такая величина называется квантиль)\\
$x_0 := -\infty$, $x_N = +\infty$\\
Пусть $x \in [x_k, x_{k+1})$, тогда\\
$$ F_n^*(x,\omega) - F(x) \leq F_n^*(x_{k+1} - 0) - F(x_k) = (F_n^*(x_{k+1} - 0) - F(x_{k+1} - 0)) + ( F(x_{k+1} - 0) - F(x_k)) $$
во второй скобке уменьшаемое $\leq \frac{k + 1}{N}$, а вычитаемое $\geq \frac{k}{N}$, значит разность $\leq \frac{1}{N}$\\
$\Rightarrow$
$$F_n^*(x,\omega) - F(x) \leq F_n^*(x_{k+1} - 0) - F(x_{k+1} - 0) + \frac{1}{N}$$
Аналогично
$$ F_n^*(x,\omega) - F(x) \geq F_n^*(x_{k}) - F(x_{k}) - \frac{1}{N} $$
Значит
$$ \forall x \in \mathbb{R}\ |F_n^*(x,\omega) - F(x)| \leq \max_{0\leq k \leq N-1, 0 \leq l \leq N-1}\{ |F_n^*(x_{k}) - F(x_{k})| + |F_n^*(x_{l+1} - 0) - F(x_{l+1} - 0)|\} + \frac{1}{N} $$
\\
ввиду сходимости эмпирической ф.р к истинной п.н. каждый член $|F_n^*(x_{k}) - F(x_{k})| + |F_n^*(x_{l+1} - 0) - F(x_{l+1} - 0)|$ стремится к нулю. Соответственно, $\forall \varepsilon > 0$ выберем такое $N$, что $\frac{1}{N} < \varepsilon$\\
Это и будет означать, что $D_n \xrightarrow{\text{п.н.}} 0$\\
$\blacktriangle$
\\


\end{document}