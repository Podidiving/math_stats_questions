	\documentclass{article}
\usepackage[utf8]{inputenc}

\title{математическая статистика. Билеты}
\author{alexander.veselyev }
\date{\today}

\usepackage{natbib}
\usepackage{graphicx}
\usepackage[utf8]{inputenc} % указывает кодировку документа
\usepackage[T2A]{fontenc} % указывает внутреннюю кодировку TeX 
\usepackage[russian]{babel} % указывает язык документа

\usepackage{amsmath}

\usepackage{amssymb}
\usepackage{amsthm}



\begin{document}
\section{Теорема об экспоненциальном семействе.}
\textbf{Теорема}\\
Пусть X - наблюдение с распределением $P\in\mathcal{P}=\{P_\theta:\ \theta\in\Theta\subset\mathbb{R}^k\}$\\
$p_\theta(x) = h(x)exp(\sum_{i=1}^k{a_i(\theta)S_i(x)} + b(\theta))$\\
Тогда $(S_1(X),\dots,S_k(X))$ - полная достаточная статистика, если\\ $(a_1,\dots,a_k)(\Theta) \supset$ k-мерный параллелепипед.\\
$\blacktriangleleft$\\
Функция правдоподобия: \\$f_\theta(X_1,\dots,X_n) = h(X_1)\dots h(X_n)exp(\sum_{i=1}^n{\langle a(\theta), S(X_i)\rangle} + nb(\theta))$\\
$\Rightarrow$ достаточность следует из критерия Неймана-Фишера\\
Докажем полноту.\\
$\varphi = \varphi^+ - \varphi^-$
$$\mathbf{E}_\theta\varphi(S(X)) = \int_{\mathbb{R}^n}\varphi(S(x))h(x_1)\dots h(x_n)exp(\langle a(\theta), S(x)\rangle + nb(\theta))dP_\theta = 0$$
(обозначим $H(X) = h(x_1)\dots h(x_n)$)\\
$\Rightarrow$
$$ \int_{\mathbb{R}^n}{\varphi^+(S(x))H(x)exp(\langle a(\theta), S(x)\rangle)dP_\theta} = \int_{\mathbb{R}^n}{\varphi^-(S(x))H(x)exp(\langle a(\theta), S(x)\rangle)dP_\theta}  $$
Определим меру $\nu$ на $\mathbb{R}^k$: $\forall B \in \mathcal{B}(\mathbb{R}^k):$\\
$\nu(B) = \int_{x: S(x)\in B}{H(x)dP_\theta} $\\
Эта мера будет $\sigma$-конечной\\
\textit{Напоминание} $\sigma$-конечная мера --- такая мера, что все пространство может быть представлено в виде счетного объединения измеримых множеств конечной меры.\\
$$ \int_{\mathbb{R}^k}{\varphi^+(S)exp(\langle a(\theta), S \rangle)d\nu}  \int_{\mathbb{R}^k}{\varphi^-(S)exp(\langle a(\theta), S \rangle)d\nu}$$
(т.е. мы сделали замену переменных в интеграле Лебега)\\
\textbf{Лемма}(б/д)\\
Пусть $\mu_1,\ \mu_2$ - $\sigma$-конечные меры на $\mathcal{B}(\mathbb{R}^k)$ и $\forall a$ из некоторого $k$-мерного параллелепипеда:
$$ \int_{\mathbb{R}^k}{exp(\langle a, s  \rangle)d\mu_1} = \int_{\mathbb{R}^k}{exp(\langle a, s  \rangle)d\mu_2} $$
Тогда $\mu_1 = \mu_2$\\
(Продолжение док-ва)\\
Пусть $\nu^+(B) = \int_B{\varphi^+(S)d\nu}$, $\nu^-(B) = \int_B{\varphi^-(S)d\nu}$\\
Эти меры, опять же, $\sigma$-конечные. Т.е. имеем:
$$ \int_{\mathbb{R}^k}{exp(\langle a(\theta), s  \rangle)d\nu^+} = \int_{\mathbb{R}^k}{exp(\langle a(\theta), s  \rangle)d\nu^-}$$
$\Rightarrow$\\
$\nu^+ = \nu^-$ $\Rightarrow$ $\varphi^+ = \varphi^-$ $\nu$-почти всюду\\
$$ 0 = \int_{S: \varphi^+(S) \neq \varphi^-(S)}{d\nu} = \nu(\{ S:\ \varphi^+(S)\neq\varphi^-(S)  \}) = \int_{x: \varphi^+(S(x))\neq\varphi^-(S(x))}{H(x)dP_\theta}$$
$H(x)>0$ (Если $H(x) = 0$, то изначально вместо $\mathbb{R}^n$ можно будет взять другой носитель)\\
$\Rightarrow$
$$ \int_{x: \varphi^+(S(x))\neq\varphi^-(S(x))}{dP_\theta} = 0 $$
$\Rightarrow$
$$ P_\theta(S: \varphi^+(S) \neq \varphi^-(S)) = 0\ \ \forall\theta\in\Theta $$
$\Rightarrow\ \ \varphi = 0$ $P_\theta$ п.н. $\forall\theta\in\Theta$ $\Rightarrow$ $S(X)$ - полная\\
$\blacktriangleright$\\

\end{document}