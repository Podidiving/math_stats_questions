\documentclass[25pt]{article}
\usepackage[utf8]{inputenc}

\title{математическая статистика. Билеты}
\author{alexander.veselyev }
\date{\today}

\usepackage{natbib}
\usepackage{graphicx}
\usepackage[utf8]{inputenc} % указывает кодировку документа
\usepackage[T2A]{fontenc} % указывает внутреннюю кодировку TeX 
\usepackage[russian]{babel} % указывает язык документа

\usepackage{amsmath}

\usepackage{amssymb}
\usepackage{amsthm}

\newcommand\independent{\protect\mathpalette{\protect\independenT}{\perp}}
\def\independenT#1#2{\mathrel{\rlap{$#1#2$}\mkern2mu{#1#2}}}


\usepackage{color}   %May be necessary if you want to color links
\usepackage{hyperref}
\hypersetup{
    colorlinks=true, %set true if you want colored links
    linktoc=all,     %set to all if you want both sections and subsections linked
    linkcolor=blue,  %choose some color if you want links to stand out
}

\begin{document}


\section{Свойства оценки максимального правдоподобия: экстремальное свойство и состоятельность.}

\textbf{Условия Регулярности модели}\\
1) $\{P_\theta,\ \theta\in\Theta\}$ - семейство распределений, доминируемое относительно
меры $\mu$, $P_{\theta_1} \neq P_{\theta_2}$ при $\theta_1 \neq \theta_2$\\
2) $\mathcal{A} = \{x\in\mathcal{X},\ p_\theta(x)>0\}$ не зависит от $\theta$\\
3) Наблюдение $X$ - выбока из $P \in \{P_\theta,\ \theta\in\Theta\}$\\
\textbf{Теорема} (экстремальное свойство правдоподобия)\\
В условиях регулярности 1-3 $\forall \theta_0,\ \theta\in\Theta,\ \theta_0\neq\theta$ выполнено:\\
$P_{\theta_0}(f_{\theta_0}(X_1\dots X_n)>f_\theta(X_1\dots X_n)) \xrightarrow[n\rightarrow+\infty]{} 1$\\
$\blacktriangle$
\\
$f_{\theta_0}(X_1\dots X_n)>f_\theta(X_1\dots X_n)\ \Leftrightarrow\ \frac{1}{n}\ln\frac{f_\theta(X)}{f_{\theta_0}(X)}<0$\\
т.к. $f_\theta(X) = \prod_{i=1}^n p_\theta(X_i)\ \Rightarrow$ по УЗБЧ\\
$\frac{1}{n}\ln\frac{f_\theta(X)}{f_{\theta_0}(X)}=\frac{1}{n}\sum_{i=1}^n{\ln\frac{p_\theta(X_i)}{p_{\theta_0}(X_i)}}\xrightarrow{P_\theta\text{п.н.}}\textbf{E}_{\theta_0}\ln\frac{p_\theta(X_1)}{p_{\theta_0}(X_1)}$\\
Тогда $P_{\theta_0}(f_{\theta_0}(X_1\dots X_n)>f_\theta(X_1\dots X_n))\rightarrow P_{\theta_0}(\textbf{E}_{\theta_0}\ln\frac{p_\theta(X_1)}{p_{\theta_0}(X_1)} < 0)$\\
Вероятность справа 0 или 1. Покажем, что 1\\
$$\textbf{E}_{\theta_0}\ln\frac{p_\theta(X_1)}{p_{\theta_0}(X_1)} = \int_{\mathcal{A}}\ln\frac{p_\theta(x)}{p_{\theta_0}(x)}p_{\theta_0}(x)\mu(d x)$$
$\ln\frac{p_\theta(x)}{p_{\theta_0}(x)} = \ln(1 + \frac{p_\theta(x)}{p_{\theta_0}(x)} - 1)$ и воспользуемся
тем, что\\ $\ln(1+x)\leq x,\ x>-1$
$$\int_{\mathcal{A}}\ln\frac{p_\theta(x)}{p_{\theta_0}(x)}p_{\theta_0}(x)\mu(d x) $$
$$\leq \int_{\mathcal{A}}\left(\frac{p_\theta(x)}{p_{\theta_0}(x)} - 1\right)p_{\theta_0}(x)\mu(d x) = \int_\mathcal{A}{p_\theta(x)\mu(d x)} -\int_\mathcal{A}{p_{\theta_0}\mu(d x)} = 0$$
$$\int_\mathcal{A}{p_\theta(x)\mu(d x)} -\int_\mathcal{A}{p_{\theta_0}\mu(d x)} = 0$$
т.к. это плотности, и интегралы равны 1\\
Когда возможно равенство?\\
только когда $\frac{p_\theta(x)}{p_{\theta_0}(x)} = 1$, что противоречит 1му условию регулярности.\\
$\Rightarrow$ неравенство стогое    
\\
$\blacktriangle$
\\
\textbf{Условия регулярности} (продолжение)
\\
4) $\Theta$ - окрытый интервал на $\mathbb{R}$\\
5) Плотность $p_\theta(x)$ - непрерывно дифференцируема по $\theta$ при $x\in\mathcal{A}$\\
\textbf{Теорема}\\
В условиях регулярности, если уравнение $\frac{\partial}{\partial\theta}f_\theta = 0$ имеет решение, то одно из его решений является состоятельной оценкой $\theta$\\
$\blacktriangleleft$\\
$\theta_0\in\Theta$, $\varepsilon>0$ т.ч. $(\theta_0 -\varepsilon, \theta_0 + \varepsilon) \subset\Theta$\\
$$P_{\theta_0}\left( f_{\theta_0}(X_1,\dots,X_n) > \max\{ f_{\theta_0 - \varepsilon}(X_1,\dots,X_n),f_{\theta_0 + \varepsilon}(X_1,\dots,X_n) ) \}  \right) \rightarrow 1$$
$$P_{\theta_0}\left( \text{на} (\theta_0 -\varepsilon, \theta_0 + \varepsilon) \ \exists\text{ решение уравнения } \frac{\partial}{\partial\theta}f_\theta = 0    \right) \rightarrow 1 $$
$\Rightarrow$
$$ P_{\theta_0}\left(\text{ ближайшее к } \theta_0 \text{ решение лежит в } (\theta_0 - \varepsilon, \theta_0 + \varepsilon)\right) \rightarrow 1 $$
Пусть $\theta^*$ ближайшее к $\theta_0$ решение $\Rightarrow$ $\theta^*\xrightarrow{P_{\theta_0}}\theta_0$\\
$\blacktriangleright$\\

\textbf{Следствие} (состоятельность ОМП)\\
Пусть в условиях предыдущей теоремы $\forall n$ $\forall X_1,\dots,X_n$ $\exists!$ решение $\theta_n^*$ уравнения правдоподобия.\\
Тогда $\theta^*_n$ - состоятельная оценка параметра $\theta$ и с вероятностью, стремящейся к 1, является ОМП. (т.е. решение уравнения правдоподобия - максимум)\\
$\blacktriangle$
\\
Состоятельность следует из предыдущей теоремы.\\
Пусть $\theta$ - истинное значение, $[\theta-a,\theta+a]\subset\Theta$
$\Rightarrow$ с вероятностью, стремящейся к 1, $f_\theta(X)$ достигает максимума на отрезке $[\theta-a,\theta+a]$ в точке $\theta^*_n$, а других корней правдоподобия нет.\\
Если $\Hat{\theta}$ - ОМП (в ней не обязательно $\frac{\partial}{\partial\theta}\ln{f_\theta(x)}=0$), не совпадающая с $\theta^*_n$, то где-то между ними - точка минимума, в которой $\frac{\partial}{\partial\theta}\ln{f_\theta(x)}=0$, что противоречит условию единственности решения уравнения.\\
$\blacktriangle$
\\

\end{document}