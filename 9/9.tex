\documentclass{article}
\usepackage[utf8]{inputenc}

\title{математическая статистика. Билеты}
\author{alexander.veselyev }
\date{\today}

\usepackage{natbib}
\usepackage{graphicx}
\usepackage[utf8]{inputenc} % указывает кодировку документа
\usepackage[T2A]{fontenc} % указывает внутреннюю кодировку TeX 
\usepackage[russian]{babel} % указывает язык документа

\usepackage{amsmath}

\usepackage{amssymb}
\usepackage{amsthm}


\begin{document}
\section{Эффективность и асимптотическая эффективность оценки максимального правдоподобия.}
\textbf{Условия регулярности} (продолжение) для ОМП\\
4) $p_\theta$ дифференцируема на $\Theta$\\
5) Плотность $p_\theta(x)$ трижды непрерывно дифференцируема на $\Theta$ для $\forall x\in\mathcal{A}$\\
6) Интеграл $\int_\mathcal{A}{p_\theta(x)\mu(dx)}$ можно трижды непрерывно дифференцировать под знаком интеграла\\
7) Информация Фишера $i(\theta) = \mathbf{E}_\theta\left(\frac{\partial}{\partial\theta}\ln{p_\theta(x)}\right)^2$ одного наблюдения\\
8) $\forall\theta_0\in\Theta\ \exists c>0$ и $H(x)$ т.ч. $\forall x\in\mathcal{A} \forall \theta\in(\theta_0-c;\theta_0+c)$
$$ \left| \frac{\partial^3}{\partial\theta^3}\ln{p_\theta(X)}\right| \leq H(X) $$
и $\mathbf{E}_\theta H(X_1) < +\infty$\\
\textbf{Теорема} (об асимптотической нормальности ОМП)\\
В условиях регулярности 1-8\\
1) Пусть $\widehat{\theta}$ - состоятельная последовательность решений уравнения правдоподобия. Тогда $\widehat{\theta}$ - асимптотически нормальная оценка $\theta$ с асимптотической дисперсией $\frac{1}{i(\theta)}$ \\
2) Пусть $\widehat{\theta}$ - какая-то асимптотически нормальноая оценка $\theta$ с асимптотической диспресией $\sigma^2(\theta)$, причем $\sigma(\theta)$ непрерывна по $\theta$. Тогда $\forall\theta
\in \Theta:\ \sigma^2(\theta) \geq \frac{1}{i(\theta)}$\\
\textbf{Вывод} в условиях регулярности 1-8 ОМП является наилучшей в асимптотическом подходе среди всех асимптотически нормальных оценок с непрерывной асимптотической дисперсией.\\
\textbf{Теоерма} (эффективность ОМП)\\
В условиях неравенства Рао-Крамера, если $\widehat{\theta}$ - эффективная оценка $\theta$, то $\widehat{\theta}$ - ОМП.\\
$\blacktriangleleft$
\\
По критерию эффективности: $\widehat{\theta} - \theta = \frac{1}{\mathcal{I}_X(\theta)}\mathcal{U}_\theta(X)$\\
$$\mathcal{U}_\theta(X) = \frac{\partial}{\partial\theta}\ln{p_\theta(X)} = \mathcal{I}_X(\theta)(\widehat{\theta} - \theta) $$
Получаем: при $\theta < \widehat{\theta}$: $\frac{\partial}{\partial\theta}\ln{p_\theta(X)} > 0$ 
(т.к. $\mathcal{I}_X(\theta) \geq 0$)\\ 
При $\theta > \widehat{\theta}$: $\frac{\partial}{\partial\theta}\ln{p_\theta(X)} < 0$\\
$\Rightarrow$ на $\widehat{\theta}$ достигается максимум функции правдоподобия $\Rightarrow$ $\widehat{\theta}$ - ОМП.\\
$\blacktriangleright$
\end{document}