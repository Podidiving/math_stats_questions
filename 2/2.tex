\documentclass[25pt]{article}
\usepackage[utf8]{inputenc}

\title{математическая статистика. Билеты}
\author{alexander.veselyev }
\date{\today}

\usepackage{natbib}
\usepackage{graphicx}
\usepackage[utf8]{inputenc} % указывает кодировку документа
\usepackage[T2A]{fontenc} % указывает внутреннюю кодировку TeX 
\usepackage[russian]{babel} % указывает язык документа

\usepackage{amsmath}

\usepackage{amssymb}
\usepackage{amsthm}

\newcommand\independent{\protect\mathpalette{\protect\independenT}{\perp}}
\def\independenT#1#2{\mathrel{\rlap{$#1#2$}\mkern2mu{#1#2}}}


\usepackage{color}   %May be necessary if you want to color links
\usepackage{hyperref}
\hypersetup{
    colorlinks=true, %set true if you want colored links
    linktoc=all,     %set to all if you want both sections and subsections linked
    linkcolor=blue,  %choose some color if you want links to stand out
}

\begin{document}


\section{Теорема о наследовании сходимостей. Лемма Слуцкого с доказательством и ее применение. Наследование асимптотической нормальности. Многомерный случай. Наследование состоятельности.}

\textbf{Напоминание}
\\
\textbf{Виды сходимостей случайных векторов}
\begin{enumerate}
    \item $\xi_n \xrightarrow{\text{п.н.}} \xi$, если $P(\omega:\ \lim \xi_n(\omega) = \xi(\omega)) = 1$
    \item  $\xi_n \xrightarrow{P} \xi$, если $\forall \varepsilon > 0 \ P(\|\xi_n - \xi\|_2 > \varepsilon) \rightarrow 0$
    \item  $\xi_n \xrightarrow{L_p} \xi$, если $\mathbf{E}(\|\xi_n-\xi\|_p)^p \rightarrow 0 $
    \item  $\xi_n \xrightarrow{d} \xi$, если для любой непрерывной ограниченной функции $f$: 
    $\mathbf{E}(f(\xi_n)) \rightarrow \textbf{E}(f(\xi))$
\end{enumerate}
\textbf{Взаимосвязь} \\
Из сходимости почти наверное следует сходимость по вероятности
\\
Из сходимости в $L_p$ следует сходимость по вероятности
\\
Из сходимости по вероятности следует сходимость по распределению
\\ \\ \\ 

\textbf{Теорема}(наследование сходимостей) \\
1) Пусть $\xi_n \xrightarrow{\text{п.н.}} \xi$  - случайные векторы размера m\\
Пусть $h : \mathbb{R}^m \rightarrow \mathbb{R}^k$ - функция, непрерывная почти всюду относительно
распределения $\xi$ (т.е. $\exists B \in \mathcal{B}(\mathbb{R}^m)$ т.ч. $P(\xi \in B) = 1$ и $h$ непрерывна на $B$.\\
Тогда $h(\xi_n) \xrightarrow{\text{п.н.}} h(\xi)$
\\
2) Аналогичное верно, если заменить сходимость п.н. на сходимость по вероятности\\
3) Пусть $\xi_n \rightarrow \xi$ по распределению - случайные векторы размера m\\
Пусть $h : \mathbb{R}^m \rightarrow \mathbb{R}^k$ - функция, непрерывная на множестве значений $\xi$\\
Тогда $h(\xi_n) \xrightarrow{d} h(\xi)$.
\\
$\blacktriangle$
\\
1) $P(\lim h(\xi_n) = h(\xi)) \geq P(\{\lim\xi_n = \xi\} \cap B) = 1$ (т.к. справа такие точки, что они принадлежат B и на них сходится последовательность сл. в. из непрерывности h на B будет следовать сходимость. Пересечение событий с вероятностью 1 дает вероятность 1)\\
2) Предположим, что это не так. Тогда существует подпоследовательность $\{\xi_{n_m}\}$ т.ч. $\exists \varepsilon > 0 \ \exists \varepsilon_1 > 0$ :
$$ P(\|h(\xi_{n_m}) - h(\xi)\|_2 > \varepsilon) > \varepsilon_1 $$
Из последовательности $\xi_{n_m} \rightarrow \xi$ сходящуюся по вероятности можно выбрать подпоследовательность $\{ \xi_{n_{m_r}}\}$ сходящуюся п.н. $\Rightarrow$ $h(\xi_{n_{m_r}}) \xrightarrow{\text{п.н.}} h(\xi)$ по пункту 1. Противоречие
\\
3) $\xi_n \xrightarrow{d} \xi$ $\Leftrightarrow$ $\forall f$ - непрерывной и ограниченной $\mathbf{E}(f(\xi_n)) \rightarrow \mathbf{E}(f(\xi))$ $\Rightarrow$ $\forall g$ $\mathbf{E}(g(h(\xi_n))) \rightarrow \mathbf{E}(g(h(\xi)))$ т.к. $g(h)$ - непрерывная и ограниченная\\
$\blacktriangle$\\ \\

\textbf{Лемма}(Слуцкого) \\
Пусть последовательности случайных величин $\{\xi_n\}$ и $\{\eta_n\}$ таковы, что
$\xi_n \xrightarrow{d} \xi$ и $\eta_n \xrightarrow{d} C$\\
Тогда $\xi_n + \eta_n \xrightarrow{d} \xi + C$ и $\xi_n\eta_n \xrightarrow{d} \xi C$
\\
$\blacktriangle$
\\
(В доказательстве используется другое, эквивалентное определение сходимости по распределению: последовательность функций распределения стремится к предельной функции распределения для любой точки непрерывности предельной функции распределения)
\\
Пусть $t$ - точка непрерывности $F_{\xi + \eta}$ пусть $F_{\xi + \eta}$ непрерына в точках $t \pm \varepsilon$
$$ F_{\xi_n + \eta_n}(t) = P(\xi_n + \eta_n \leq t) = P(\xi_n + \eta_n \leq t,\ \eta_n < C - \varepsilon) + P(\xi_n + \eta_n \leq t,\ \eta_n \geq C - \varepsilon) $$ $$ \leq P(\eta_n < C) + P(\xi_n \leq t - C + \varepsilon) = P(\eta_n < C) + P(\xi_n + C \leq t + \varepsilon)$$
Первая вероятность стремится к 0 (сходимость по распределению к константе влечет сходимость по вероятности)
\\
Из теоремы о наследовании: $\xi_n + C \rightarrow \xi + C$ 
\\
$\Rightarrow$ $P(\xi_n + C \leq t + \varepsilon) = F_{\xi_n + C}(t + \varepsilon) \rightarrow F_{\xi + C}(t + \varepsilon)$
\\
$\Rightarrow$ $P(\eta_n < C) + P(\xi_n + C \leq t + \varepsilon) \rightarrow F_{\xi + C}(t + \varepsilon)$
\\
$$ F_{\xi_n + \eta_n}(t) = P(\xi_n + \eta_n \leq t) = P(\xi_n + \eta_n \leq t,\ \eta_n > C + \varepsilon) + P(\xi_n + \eta_n \leq t,\ \eta_n \leq C + \varepsilon)$$
$$ \geq  P(\xi_n + \eta_n \leq t,\ \eta_n \leq C + \varepsilon) \geq  P(\xi_n + C + \varepsilon \leq t,\ \eta_n \leq C + \varepsilon)$$
$$ =  P(\xi_n + C + \varepsilon \leq t) - P(\xi_n + C + \varepsilon \leq t,\ \eta_n > C + \varepsilon) $$
$$ \geq P(\xi_n + C + \varepsilon \leq t) - P(\eta_n > C + \varepsilon)$$
Второе слагаемое, аналогично, стремится к 0. первое слагаемое (= $F_{\xi_n + C}(t - \varepsilon)$) аналогично стремится к $F_{\xi + C}(t - \varepsilon)$
\\
Значит: 
$$ F_{\xi + C}(t-\varepsilon) \leq \underline{\lim} F_{\xi_n + \eta_n} \leq \overline{\lim} F_{\xi_n + \eta_n} \leq F_{\xi + C}(t + \varepsilon) $$
\\
Получается $\lim F_{\xi_n + \eta_n}(t) = F_{\xi + \eta}(t)$
\\
Доказательство для произведения аналогично.\\
$\blacktriangle$
\\
Применение леммы Слуцкого: доказательство факта, что из ассимптотической нормальности следует сильная состоятельность (билет 1)
\\ \\ 
\textbf{Теорема}($\delta$-метод)\\
Пусть $\xi_n \xrightarrow{d} \xi$, числовая последовательность $b_n \rightarrow 0$, $b_n \neq 0$, функция $h: \mathbb{R} \rightarrow \mathbb{R}$ - дифференцируема в точке $a$. Тогда
$$ \frac{h(a + \xi_n b_n) - h(a)}{b_n} \xrightarrow{d} h'(a)\xi$$ 
\\
$\blacktriangle$
\\
Определим функцию $H(x) = \frac{h(a + x) - h(a)}{x}$ и доопределим в 0 значением $h'(a)$. $H(x)$ непрерына в точке 0.
\\
По лемме Слуцкого $\xi_n b_n \xrightarrow{d} 0 \Rightarrow \xi_n b_n \xrightarrow{P} 0$
\\
Воспользуемся теоремой о наследовании сходимостей:
$$ \frac{h(a + \xi_n b_n) - h(a)}{\xi_n b_n} = H(\xi_n b_n) \xrightarrow{P} H(0) = h'(a) $$
Значит, по лемме Слуцкого:
$$ \frac{h(a + \xi_n b_n) - h(a)}{b_n} \xrightarrow{d} h'(a)\xi $$
\\
$\blacktriangle$
\\
\textbf{Следствие}(наследование ассимптотической нормальности)\\
Пусть $\theta^*$ - асимптотически нормальная оценка $\theta$ с асимптотической дисперсией $\sigma^2(\theta)$,
функция $\tau$ непрерывно дифференцируема на $\Theta \subset \mathbb{R}$.\\
Тогда $\tau(\theta^*)$ - асимптотически нормальная оценка параметра $\tau(\theta)$ с асимптотической дисперсией $\sigma^2(\theta)[\tau'(\theta)]^2$
\\
$\blacktriangle$
\\
$\xi_n = \sqrt{n}(\theta^* - \theta),\ b_n = \frac{1}{\sqrt{n}},\ a = \theta,\ h = \tau$\\
Осталось воспользоваться определением асимптотической нормальности и $\delta$-методом.
\\
$\blacktriangle$
\\
\textbf{Утверждение}\\
$\delta$-метод работает и в многомерном случае: производная заменяется на матрицу частных производных.\\
\textbf{Следствие}(многомерный случай)\\
В многомерном случае на функцию $\tau$ накладывается ограничение на существование матрицы частных производных в каждой точке $\theta \in \Theta$. Тогда, если $\theta^*$ - асимптотически нормальная оценка $\theta$ с асимптотической матрицей ковариации $\Sigma(\theta)$, то
$\tau(\theta^*)$ - асимптотически нормальная оценка $\tau(\theta)$ с асимптотической матрицей ковариаций $(\tau'(\theta))^T\Sigma(\theta)\tau'(\theta)$
\\
\\
\textbf{Утверждение}(наследование состоятельности)\\
Если $\theta^*$ состоятельная (сильно состоятельная) оценка $\theta$ и функция $\tau$ непрерывна во всех точках $\theta \in \Theta$, то $\tau(\theta^*)$ - состоятельная (сильно состоятельная) оценка $\tau(\theta)$
\\
$\blacktriangle$
\\
Утверждение - тривиальное следствие из теоремы о наследовании сходимостей.
\\
$\blacktriangle$
\\

\end{document}